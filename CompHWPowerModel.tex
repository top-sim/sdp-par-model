\documentclass[useAMS,usenatbib,referee]{article}
\usepackage{amsmath}
\usepackage{microtype}
\usepackage{mathptmx}
\usepackage{graphicx}
\usepackage{booktabs}
\usepackage{captcont}
\usepackage[colorlinks=true,citecolor=blue]{hyperref}
\usepackage{emaxima}
\usepackage{natbib}

\newcommand{\unit}[1]{\mathrm{#1}}
\newcommand{\unitp}[2]{\ensuremath{\mathrm{#1}^{#2}}}

\title{Power Models for Compute Hardware\\
  Rev: \input{|"git describe --dirty"}}
\author{B. Nikolic\\
  Astrophysics Group, Cavendish Laboratory, Cambridge CB3 0HE, UK
  \\\url{email:b.nikolic@mrao.cam.ac.uk}
 \\\url{http://www.mrao.cam.ac.uk/~bn204/}}

\begin{document}

\maketitle

\tableofcontents

\section{Scope of this document}

This document summarises the current power budget calculations for
SDP. It is currently presents only the option for {\bf baseline
  design} with, i.e., full spectral resolution, full resolution, no
baseline-dependent averaging, and {\bf 2017 technology}. The scaling
of power requirements can be found by in reference document
\cite{MajCycleModel}. 

Important caveats:
\begin{itemize}
\item Only diagonal Muller matrix elements are considered. We are
  currently analysing which telescopes will require the off-diagonal
  elements.
\item Flat efficiency of about 0.25 (more precisely 50\% of LINPACK)
  is assumed for all operations. Ongoing analysis will provide
  figures on a firmer basis.
\end{itemize}


\section{Summary of changes vs version 0.2}

\begin{enumerate}
\item For consistency with other computations gridding is taken to
  take 8 FLOP rather than 4 FAM operations. This has the greatest
  proportional effect on the gridding intensive telescopes (SKA1-low)
\item A first approximation of the computation of convolution kernels
  is included in the estimate. It is assumed that all kernels are
  baseline-dependent.
\item Mistake in number of SKA1-Survey antennas has been corrected. 
\end{enumerate}


\section{Floating point operation power requirements}

Non-immersed systems currently get about $\sim 3.6\times 10^9$
double-precision FLOPS per W (e.g., Wilkes cluster in Cambridge,
\cite{Green500Nov2013}). This is for the whole system, actual
performance with (close-to-best-case) LINPACK tests. 

They are generally based on Kepler, 2012 technology. Assuming two year
doubling time scale, we can model efficiency approximately as:

\begin{maxima}[]
DPP(t) := 3.6e9 * 2**((t-2012)/2) ` (Ops/s/W);
float(DPP(2017));
\maximaoutput*
\m  \mathrm{DPP}\left(t\right)\mathbin{:=}3.6 \times 10^{+9}\,2^{{{t-2012}\over{2}}}\;{{\mathrm{Ops}}\over{\mathrm{W}\,\mathrm{s}}} \\
\m  2.0 \times 10^{+10}\;{{\mathrm{Ops}}\over{\mathrm{W}\,\mathrm{s}}} \\
\end{maxima}
Again note this is efficiency for the system as a whole but for a
best-case type of processing.

We use calculations from the \cite{MajCycleModel} for the compute rate
major cycle and the computation of the convolution kernel:
\begin{maxima}[]
HCSKA1Low: (1.3e15+4.3e14) `(Ops/s) * 2 * Nmajor  / epsilon;
HCSKA1Mid: (1.3e16+3.7e14) `(Ops/s)* 2 * Nmajor / epsilon;
HCSKA1Survey: (9.1e15+9.8e14) `(Ops/s)* 2 * Nmajor/  epsilon;
\maximaoutput*
\m  3.5 \times 10^{+15}\;{{\mathrm{Ops}\,\mathrm{Nmajor}}\over{\varepsilon\,\mathrm{s}}} \\
\m  2.7 \times 10^{+16}\;{{\mathrm{Ops}\,\mathrm{Nmajor}}\over{\varepsilon\,\mathrm{s}}} \\
\m  2.0 \times 10^{+16}\;{{\mathrm{Ops}\,\mathrm{Nmajor}}\over{\varepsilon\,\mathrm{s}}} \\
\end{maxima}
Here:
\begin{description}
  \item[Nmajor] is the number of major cycles 
  \item[$\epsilon$] is a further efficiency factor. In current version
    of this document we assume this is same for all operations. This
    is probably a poor approximation which needs to be revisited. 
\end{description}

For estimate 40\,kW/rack for rack estimate. Therefore current power
and rack  estimates are:
\begin{maxima}[]
SKA1LowPower: float( (HCSKA1Low / DPP(2017)) ``MW) , Nmajor=10, epsilon=0.5;
SKA1MidPower: float( (HCSKA1Mid / DPP(2017)) ``MW) , Nmajor=10, epsilon=0.5;
SKA1SurveyPower: float( (HCSKA1Survey / DPP(2017)) ``MW) , Nmajor=10, epsilon=0.5;

SKA1LowPower  / (0.04 ` MW);
SKA1MidPower  / (0.04 ` MW);
SKA1SurveyPower/ (0.04 ` MW);
\maximaoutput*
\m  3.4\;\mathrm{MW} \\
\m  26.\;\mathrm{MW} \\
\m  20.\;\mathrm{MW} \\
\m  85. \\
\m  657. \\
\m  495. \\
\end{maxima}
for Low, Mid and Survey. 


\section{Storage}

The estimate of scratch storage required is given in
\cite{SDPDataFlow}. It is 300, 140 and 190 PetaBytes for SKA1-Low,
SKA1-Mid and SKA1-Survey respectively. We assume power consumption
of 3kW/PetaByte and 6 PetaByte / rack (slightly more optimistic than
in proposal \cite{SDP-PROP-DR-001-1-ElemConc}).


\section{Network}

In initial estimate we use verbatim the number from
\cite{SDP-PROP-DR-001-1-ElemConc}. There we estimate 5 core switches
in total at 10 kW each and 200 rack switches at 300 W each. Because
these values are much smaller than the uncertainties in power required
for the floating point operations computation we do not include them
in the budget entered in the spreadsheet.

\bibliographystyle{mn2eurl} 
\bibliography{ska.bib}

\end{document}


% Local Variables:
% LaTeX-command: "latex -shell-escape"
% End:

\documentclass[useAMS,usenatbib,referee]{article}
\usepackage{amsmath}
\usepackage{microtype}
\usepackage{mathptmx}
\usepackage{graphicx}
\usepackage{booktabs}
\usepackage{captcont}
\usepackage[colorlinks=true,citecolor=blue]{hyperref}
\usepackage{emaxima}
\usepackage{natbib}

\newcommand{\unit}[1]{\mathrm{#1}}
\newcommand{\unitp}[2]{\ensuremath{\mathrm{#1}^{#2}}}

\title{Power Models for Compute Hardware\\
  Rev: \input{|"git describe --dirty"}}
\author{B. Nikolic\\
  Astrophysics Group, Cavendish Laboratory, Cambridge CB3 0HE, UK
  \\\url{email:b.nikolic@mrao.cam.ac.uk}
 \\\url{http://www.mrao.cam.ac.uk/~bn204/}}

\begin{document}

\maketitle

\tableofcontents

\section{Scope of this document}

\section{Processing hardware}

Non-immersed systems currently get about $\sim 3.6\times 10^9$
double-precision FLOPS per W (e.g., Wilkes cluster in Cambridge,
\cite{Green500Nov2013}). This is for the whole system, actual
performance with (close-to-best case) LINPACK tests.

They are generally based on Kepler, 2012 technology. Assuming two year
doubling time scale, we can model efficiency approximately as:

\begin{maxima}[]
DPP(t) := 3.6e9 * 2**((t-2012)/2) ` (Ops/s/W);
float(DPP(2017));
\maximaoutput*
\m  \mathrm{DPP}\left(t\right)\mathbin{:=}3.6 \times 10^{+9}\,2^{{{t-2012}\over{2}}}\;{{\mathrm{Ops}}\over{\mathrm{W}\,\mathrm{s}}} \\
\m  2.04 \times 10^{+10}\;{{\mathrm{Ops}}\over{\mathrm{W}\,\mathrm{s}}} \\
\end{maxima}

Using calculations from the computing requirements:
\begin{maxima}[]
HCSKA1Low: 9e14 `(Ops/s) * 2 * Nmajor  / epsilon;
HCSKA1Mid: 1e16 `(Ops/s)* 2 * Nmajor / epsilon;
HCSKA1Survey: 7e15 `(Ops/s)* 2 * Nmajor/  epsilon;
\maximaoutput*
\m  1.8 \times 10^{+15}\;{{\mathrm{Ops}\,\mathrm{Nmajor}}\over{\varepsilon\,\mathrm{s}}} \\
\m  2.0 \times 10^{+16}\;{{\mathrm{Ops}\,\mathrm{Nmajor}}\over{\varepsilon\,\mathrm{s}}} \\
\m  1.4 \times 10^{+16}\;{{\mathrm{Ops}\,\mathrm{Nmajor}}\over{\varepsilon\,\mathrm{s}}} \\
\end{maxima}

Therefore current power estimates:
\begin{maxima}[]
float( (HCSKA1Low / DPP(2017)) ``MW) , Nmajor=10, epsilon=0.5;
float( (HCSKA1Mid / DPP(2017)) ``MW) , Nmajor=10, epsilon=0.5;
float( (HCSKA1Survey / DPP(2017)) ``MW) , Nmajor=10, epsilon=0.5;
\maximaoutput*
\m  1.77\;\mathrm{MW} \\
\m  19.6\;\mathrm{MW} \\
\m  13.7\;\mathrm{MW} \\
\end{maxima}


\section{Storage}

In initial estimate we use verbatim the number from
\cite{SDP-PROP-DR-001-1-ElemConc}. This gives 100 racks (3PB each) at
18.5 kW each.

\section{Network}

In initial estimate we use verbatim the number from
\cite{SDP-PROP-DR-001-1-ElemConc}. There we estimate 5 core switches
in total at  10 kW each and 200 rack switches at 300 W each. 


\bibliographystyle{mn2eurl} 
\bibliography{ska.bib}

\end{document}


% Local Variables:
% LaTeX-command: "latex -shell-escape"
% End:

\documentclass[useAMS,usenatbib,referee]{article}
\usepackage{amsmath}
\usepackage{microtype}
\usepackage{mathptmx}
\usepackage{graphicx}
\usepackage{booktabs}
\usepackage{captcont}
\usepackage[colorlinks=true,citecolor=blue]{hyperref}
\usepackage{emaxima}
\usepackage{natbib}

\newcommand{\unit}[1]{\mathrm{#1}}
\newcommand{\unitp}[2]{\ensuremath{\mathrm{#1}^{#2}}}

\title{Data Flows in SDP\\
  Rev: \input{|"git describe --dirty"}}
\author{B. Nikolic\\
  Astrophysics Group, Cavendish Laboratory, Cambridge CB3 0HE, UK
  \\\url{email:b.nikolic@mrao.cam.ac.uk}
 \\\url{http://www.mrao.cam.ac.uk/~bn204/}}

\begin{document}

\maketitle

\tableofcontents

\section{Scope of this document}

This documents sets out initial numerical models for data rates and
flows within the SDP element. It is currently in {\bf in draft} and is
{\bf not complete}: it instead concentrates on some of the perceived
high-risk areas. 

\section{Visibilities input data rate}

The input data rate into the SDP is largely defined by the
\cite{DewdneyDD001-1} as amended by \cite{McCoolDD003}.  If we assume
8 bytes per visibility, the calculations below show the baseline
design data rates in terms of TBytes/s. 

\begin{maxima}
SKA1Low : [ tdump = 0.6 `s, Na = 1024 , Nf= 256000, Nbeam=1];
SKA1Mid : [ tdump = 0.08 `s, Na = 190+64 , Nf= 256000, Nbeam=1];
SKA1Survey: [tdump = 0.3`s, Na=94, Nf=256000, Nbeam=36];
SDPInRate : Na * (Na-1)/2 * Nf * Nbeam* 4 / tdump * (8`byte) ``(Tbyte);

calcTel(SDPInRate, SKA1Low);
calcTel(SDPInRate, SKA1Mid);
calcTel(SDPInRate, SKA1Survey);

\maximaoutput*
\m  \left[ \mathrm{tdump}=0.6\;\mathrm{s} , N_{\rm a}=1024 , N_{\rm f}=256000 , N_{\rm beam}=1 \right] \\
\m  \left[ \mathrm{tdump}=0.08\;\mathrm{s} , N_{\rm a}=254 , N_{\rm f}=256000 , N_{\rm beam}=1 \right] \\
\m  \left[ \mathrm{tdump}=0.3\;\mathrm{s} , N_{\rm a}=94 , N_{\rm f}=256000 , N_{\rm beam}=36 \right] \\
\m  {{\left(N_{\rm a}-1\right)\,N_{\rm a}\,N_{\rm beam}\,N_{\rm f}}\over{62500000000\,\mathrm{tdump}}}\;\mathrm{Tbyte} \\
\m  \mathrm{SDPInRate(SKA1Low)}=7.15\;{{\mathrm{Tbyte}}\over{\mathrm{s}}} \\
\m  \mathrm{SDPInRate(SKA1Mid)}=3.29\;{{\mathrm{Tbyte}}\over{\mathrm{s}}} \\
\m  \mathrm{SDPInRate(SKA1Survey)}=4.3\;{{\mathrm{Tbyte}}\over{\mathrm{s}}} \\
\end{maxima}


\section{``UV buffer'' rates chunk sizes}

The ``UV buffer'' is the system component that temporarily stores
visibility data until the imaging and calibration processing has
converged to a faithful representation of the sky and/or
visibility-based processing has extracted the information required
from raw visibilities. 

It is expected that some data reduction will be possible in the ingest
processing stage so that UV buffer will not normally handle the full
input data rate from the Correlator. However, until the feasibility of
this data reduction is confirmed we assume the maximal data input
rate. 

\subsection{Capacity}

In this case the input data rate in the UV buffer is just the maximum
data rate given above. Assuming maximum 12 hour duration of data
accumulation the storage requirement is :
\begin{maxima}[]
SDPUVBufSize: SDPInRate * 12* 3600 `s;
calcTel(SDPUVBufSize, SKA1Low);
calcTel(SDPUVBufSize, SKA1Mid);
calcTel(SDPUVBufSize, SKA1Survey);
\maximaoutput*
\m  {{27\,\left(N_{\rm a}-1\right)\,N_{\rm a}\,N_{\rm beam}\,N_{\rm f}}\over{39062500\,\mathrm{tdump}}}\;\mathrm{Tbyte}\,\mathrm{s} \\
\m  \mathrm{SDPUVBufSize(SKA1Low)}=308936.\;\mathrm{Tbyte} \\
\m  \mathrm{SDPUVBufSize(SKA1Mid)}=142137.\;\mathrm{Tbyte} \\
\m  \mathrm{SDPUVBufSize(SKA1Survey)}=185625.\;\mathrm{Tbyte} \\
\end{maxima}

\subsection{Organisation of input data}

Data coming out of the correlator are assumed to be organised
according to following order:
\begin{enumerate}
  \item Beam
  \item Polarisation
  \item Frequency channel
  \item Time 
  \item Baseline
\end{enumerate}
Dimensions further down the list show data which are closer in
proximity in the input data stream. 

\subsection{Natural chunks of data for imaging assuming snapshots} 

If we assume snapshot imaging technique is used then the natural chunk
of data is combination of all baselines and snapshot interval for one
beam/polarisation/frequency channel. 

Initial analysis in document \cite{MajCycleModel} shows that likely
snapshot duration is of order 100-200s for each of the telescope.
Assuming 100s for each, we get the following for the size of
basic units of data:
\begin{maxima}[]
SDPInRatePerStrm : Na * (Na-1)/2    / tdump * (8`byte) ;
SDPInUnit : (SDPInRatePerStrm * (100`s));

calcTel(SDPInUnit, SKA1Low);
calcTel(SDPInUnit, SKA1Mid);
calcTel(SDPInUnit, SKA1Survey);

\maximaoutput*
\m  {{4\,\left(N_{\rm a}-1\right)\,N_{\rm a}}\over{\mathrm{tdump}}}\;\mathrm{byte} \\
\m  {{400\,\left(N_{\rm a}-1\right)\,N_{\rm a}}\over{\mathrm{tdump}}}\;\mathrm{s}\,\mathrm{byte} \\
\m  \mathrm{SDPInUnit(SKA1Low)}=6.984 \times 10^{+8}\;\mathrm{byte} \\
\m  \mathrm{SDPInUnit(SKA1Mid)}=3.213 \times 10^{+8}\;\mathrm{byte} \\
\m  \mathrm{SDPInUnit(SKA1Survey)}=1.166 \times 10^{+7}\;\mathrm{byte} \\
\end{maxima}

It is however extremely likely that in the ingest we will:
\begin{enumerate}
  \item Re-organise so that least about 10 frequency channels are
    organised together
  \item Similarly group the four polarisation products 
\end{enumerate}
Organising 10 frequency channels together still leaves ample
data-parallelism for the processing.

Therefore it is likely that the basic chunks of UV data be:
\begin{maxima}[]
SDPUVChunk : (SDPInUnit * 4* 10);

calcTel(SDPUVChunk, SKA1Low);
calcTel(SDPUVChunk, SKA1Mid);
calcTel(SDPUVChunk, SKA1Survey);
\maximaoutput*
\m  {{16000\,\left(N_{\rm a}-1\right)\,N_{\rm a}}\over{\mathrm{tdump}}}\;\mathrm{s}\,\mathrm{byte} \\
\m  \mathrm{SDPUVChunk(SKA1Low)}=2.79 \times 10^{+10}\;\mathrm{byte} \\
\m  \mathrm{SDPUVChunk(SKA1Mid)}=1.29 \times 10^{+10}\;\mathrm{byte} \\
\m  \mathrm{SDPUVChunk(SKA1Survey)}=4.662 \times 10^{+8}\;\mathrm{byte} \\
\end{maxima}

\subsection{Shared filesystem -- number of files}

If for discussion we hypothesise:
\begin{enumerate}
  \item The UV buffer is implemented as a global shared filesystem
  \item For maximum flexibility each UV chunk is represented as an
    individual file
\end{enumerate}
then we get maximal number of:

\begin{maxima}[]
SDPNoFilesHyp: (SDPUVBufSize/SDPUVChunk)``1;

calcTel(SDPNoFilesHyp, SKA1Low);
calcTel(SDPNoFilesHyp, SKA1Mid);
calcTel(SDPNoFilesHyp, SKA1Survey);
\maximaoutput*
\m  {{216\,N_{\rm beam}\,N_{\rm f}}\over{5}} \\
\m  \mathrm{SDPNoFilesHyp(SKA1Low)}=1.106 \times 10^{+7} \\
\m  \mathrm{SDPNoFilesHyp(SKA1Mid)}=1.106 \times 10^{+7} \\
\m  \mathrm{SDPNoFilesHyp(SKA1Survey)}=3.981 \times 10^{+8} \\
\end{maxima}





\bibliographystyle{mn2eurl} 
\bibliography{ska.bib}


\end{document}

% Local Variables:
% LaTeX-command: "latex -shell-escape"
% End:
